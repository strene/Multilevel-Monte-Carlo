\def\name{Concluding Remarks}

\begin{frame}{\name{}}

    \textbf{A few pitfalls and shortcomings}\vskip2mm
    \begin{squarelist}
        \item<2-> Convergence theory with conditions in terms of unknown quantities
        \begin{circlelist}
            \item Number of samples needed to approximate $\vara_\ell$, $\cost_\ell$ is problem-dependent
        \end{circlelist}
        \item<3-> Use upscaling with care
        \begin{circlelist}
            \item Coarsest level $\ell = 0$ should have cell diameter $h \sim$ correlation length $\lambda$
        \end{circlelist}
        \item<4-> Very challenging to upscale complex models in a meaningful way
        \begin{circlelist}
            \item Channelized reservoirs, different rock types, multiphase, etc. \\
            ... and the best methods are expensive
        \end{circlelist}
        \item<5-> Not all choices of $\un$ are appropriate!
        \begin{circlelist}
            % \only<3-5>{%
            %     \item<4-5> Example: saturation at specific point and time
            %     \item<5-5> Example: binary output $\un \in \{0,1\}$ (e.g., fracture propagation)
            %     \begin{equation*}
            %         \un_\ell - \un_{\ell-1} = 
            %         \begin{cases}
            %             1 & \text{probability $p$} \\
            %             -1 & \text{probability $q$} \\
            %             0 & \text{probability $1-(p+q)$}
            %         \end{cases}
            %         \quad p,q \ll 1 \rightarrow \expect[\un_\ell - \un_{\ell-1}] \approx 0
            %     \end{equation*}
            % }%
            \item Rule of thumb: average of $\un$ should ''make sense'' for the problem at hand
        \end{circlelist}
    \end{squarelist}
    
\end{frame}

\begin{frame}{\name{}}

    \textbf{Can we do better?}\vskip2mm
    \begin{squarelist}
        \item<2-> \emph{Level} does not necessarily mean spatial resolution!
        \begin{circlelist}
            \item Solver-based: use a more accurate solver for higher levels\\
            e.g., increasing accuracy of spatial/temporal discretization with level
            \item Multiscale methods (compromise between upscaling and solver accuracy)
        \end{circlelist}
        \item<3-> MLMC does not require a geometric sequence of levels 
        \begin{circlelist}
            \item Sufficient that accuracy and cost \emph{increase} and variance \emph{decrease} with $\ell$
        \end{circlelist}
        \item<4-> Multi-index Monte Carlo -- change multiple aspects of simulation with
        level
        \begin{circlelist}
            \item Example: resolution in space \emph{and} time, $\ell \rightarrow \vect \ell = (\ell_{\vx}, \ell_t)$
        \end{circlelist}
        % \item<7-> Richardson extrapolation -- more accurate estimates of $\un_\ell - \un_{\ell-1}$
    \end{squarelist}
    
\end{frame}

\def\name{Suggested Literature}

\begin{frame}{\name{}}
	\begin{scriptsize}
    	\bibliography{refs}
	\end{scriptsize}
\end{frame}

\begin{frame}
    \vspace{1em}
    \begin{block}{}
        \small
        Developed by nuclear physicist Stanislaw Ulam during the Manhattan Project in the late 1940's\vskip5mm
        {\itshape It was at that time that I suggested an obvious name for the statistical method -- a suggestion not unrelated to the fact that Stan had an uncle who would borrow money from relatives because he "just had to go to Monte Carlo"}\vskip5mm
        \hspace*\fill{\small--- Nicholas Metropolis, {\itshape The Beginning of the Monte Carlo Method} (1987)}
    \end{block}
\end{frame}